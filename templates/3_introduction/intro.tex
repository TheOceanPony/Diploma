\chapter*{Вступ}
\addcontentsline{toc}{chapter}{Вступ}

	Задача бінокулярного стереозору --- одна з актуальних проблем у сфері комп'ютерного зору. Це метод оцінки відстані від камери до об'єктів шляхом порівняння пари їх знімків, зроблених під різними кутами.
	
	Від доповненої реальності до автономної навігації --- задачі стереозору мають багато застосувань в сучасному світі. Будучи менш точними, ніж лазерне сканування, методи стерео зору часто використовуються для отримання топографічних мап через те, що дозволяють охопити одним знімком великий регіон земної поверхні. 

	В цій роботі розглядаються алгоритми пошуку одновимірної карти зсувів, що на вхід приймають одновимірне ректифіковане зображення. Такі алгоритми дають непогане уявлення про глибину сцени (рис. 1\ref{results}) ?(while not being computationaly complex)?.
%Depth map result
\begin{figure}
	\centering
	\includegraphics[scale = 0.4]{resultdisp}
	\caption{Результат роботи одновимірного алгоритму}
	\label{results}
\end{figure}
	
\textbf{Актуальність роботи}. Offline алгоритми розв'язання задачі стереозору можуть почати свою роботу тільки після надходження всіх даних. Проте, з розвитком мережевих технологій все частіше трапляється, що необхідна інформація  завантажується безпосередньо з інтернету по мірі необхідності, а не зберігається локально. Також можлива ситуація, коли дані для обрахунку відправляються до обчислювального центру. Сучасні дата-центри мають дуже швидкі та надійні канали доступу до мережі, цього не можна сказати про їх клієнтів. Тож до часу роботи самого алгоритму буде додаватися ще й час надходження всіх даних. Цю проблему можна вирішити, використовуючи алгоритми, які можуть починати обрахунки вже після надходження першої частини даних, зменшуючи тим самим сумарний час обробки даних.

\textit{Об'єкт дослідження} --- прихована модель Маркова.

\textit{Предмет дослідження} --- ефективна оцінка прихованих параметрів моделі Маркова.

\textbf{Мета дослідження}. Розробка онлайн алгоритму для задачі стереозору.

\textbf{Практичне значення результатів}. Розроблений алгоритм можна використовувати для ефективного вирішення задачі стереозору в умовах повільного надходження даних.
	
