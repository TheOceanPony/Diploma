\chapter*{Вступ}
\addcontentsline{toc}{chapter}{Вступ}

\section{Предисловие}
	Нахождение карты глубины сцены является однии из важнейших задач в области компьютерного зрения. Решение этой задачи может осуществляться несколкими способами. Хоть они все и разные, но принцип у всех один и тот же -- чем дальше объект, тем меньше "изменения".   
	
Глаз = любой зрительный сенсор (человеческий глаз/ камера и т.п) 

\section{Способы восприяти глубины}
	Восприятие глубины (восприятие расстояния) — зрительная способность воспринимать действительность в её трёх измерениях, воспринимать расстояние до объекта.
	
	
	 Данное восприятие формируется при помощи множества так называемых "глубинных признаков". Это такие характеристики, которые у одного и того же объекта на разных расстояних -- различны.  Их можно разделить на Монокулярные -- те, для обнаружения которых будет достаточно и одного глаза / сенсора, и Бинокулярные -- полноценное восприятие которых возможно только как агрегация сенсорной информации поступающей с двух глаз. Так же, иногда их делят на Динамические -- требующие движения глаза или самого объекта, и Стационарные	-- для которых подобные действия не нужны. 
	
	\newpage
	\subsection{Монокулярные глубинные признаки}  
	\begin{itemize}
	
	\item Параллакс\\
		Один из самых знакомых нам признаков, и очень широко используемый эффект в задачах нахождения карты глубины. Суть его проста -- чем дальше объект, тем меньше он будет смещаться (на проекции) при движении глаза. 
		$$Рис. Parallax.png$$
		
	\item optical expansion!! перевести\\
		Если объект движется на/от нас, размер его проекции будет увеличиваться/ уменьшатся.
		$$Рис. Optical_expansion.png$$	
		
	\item Относительные размеры\\
		Если известно, что несколько объектов имеют похожие размеры (напр. люди), причём абсолютные размеры могут быть неизвестены, то на основании размеров их проекций можно сделать выводы про относительное расстояние до них (Если один человек на снимке значительно больше остальных, значит, скорее-всего, оно находится ближе всего к камере).
		$$Рис. Relative_sizes.png$$
		
	\item Градиент текстуры\\
		Хорошо различимые мелкие детали текстуры поверхности становятся всё хуже заметными с расстоянием. 
		$$Рис. Texture_Gradient	.png$$
		
	\item Свет и тени\\
		Тени отбрасываемые объектами, то как свет падает и отражается от объектов, позволяют определять форму объектов и их положение в пространстве.
		$$Рис. Light_Shadows.png\\siluets, contours, shadows
		$$
	\end{itemize}
	
	
	\subsection{Бинокулярные глубинные признаки}  
	\begin{itemize}
	
	\item Бинокулярный параллакс\\
		При использовании двух изображений одной и той же сцены, полученных под разными углами, можно триангулировать расстояние до объекта с высокой степенью точности. Именно на этом принципе основано 3д-кино и автостереограммы.
		$$Рис. Binocular_Parallax.png$$	
		
	\item  Конвергенция\\
		Присущий живым организмам признак, конвергенция -- кинестические ощущения от сокращения глазных мышц при фокусировке на каком-либо объекте.
		$$Рис. Convergence.png$$	
		
	\end{itemize}
	
	Это далеко не все существующие глубинные признаки. Более полный список можно найти здесь: \TeX \cite{Wikipedia:Depth_perception}
	
	В данной работе будет рассматриваться использование только бинокулярного параллакса. И хоть этот метод сам по себе не даёт абсолютной информации о глубине сцены, зная некоторые дополнительные параметры её можно получить.
	
	
	
	

	
