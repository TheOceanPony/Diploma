\chapter*{Вступ}
\addcontentsline{toc}{chapter}{Вступ}

	Задача бінокулярного стерео зору --- одна з актуальних проблем у сфері комп'ютерного зору. Це метод отримання тривимірної інформації про об'єкти, шляхом порівняння пари їх знімків зроблених під різними кутами. Базуючись на відносному розташуванні об'єктів на знімках можна зробити висновки про їх віддаленість від камер. 
	
	Від автономної навігації та фотограмметрії до доповненої реальності --– задачі стерео-зору мають багато застосувань в сучасному світі. Являючись менш точними ніж, скажемо, лазерне сканування, методи стерео зору часто використовуються для отримання топографічних мап, через те, що дозволяють охопити одним знімком великий регіон земної поверхні. 

	В цій роботі розглядаються одновимирні алгоритми пошуку карти зсувів. Це означає, що кожен рядок зображення ми опрацьовуємо незалежно від інших. Такий підхід вимагає попередньої ректифiкацiї зображення, проте є більш швидким та дає непогані результати (рис. 1).
%Depth map result
\begin{figure}
	\centering
	\includegraphics[scale = 0.2]{disp}
	\caption{Результат роботи одновимірного алгоритму}
	\label{results}
\end{figure}
	
\textbf{Актуальність роботи}. "Оффлайн"(!) алгоритми розв'язання задачі стерео зору можуть почати свою роботу тільки після надходження всіх даних. Проте, з розвитком мережевих технологій все частіше трапляється, що необхідна інформація  завантажується безпосередньо з інтернету по мірі необхідності (on demand)(!), а не зберігається локально. Також можлива ситуація, коли дані для обрахунку відправляються до обчислювального центру. Та хоч сучасні дата-центри мають дуже швидкі та надійні канали доступу до мережі, цього не можна сказати про їх "клієнтів". Тож часто до часу роботи самого алгоритму буде додаватися ще й час затримки надходження всіх даних. Цю проблему можна вирішити використовуючи алгоритми, які б могли починати обрахунки вже після надходження першої частини даних, зменшуючи тим самим сумарний час обробки даних.

\textit{Об'єкт дослідження} --- прихована модель Маркова.

\textit{Предмет дослідження} --- ефективна оцінка прихованих параметрів (моделі Маркова)?

\textbf{Мета дослідження}. Розробка онлайн алгоритму для задачі стерео зору.

\textbf{Практичне значення результатів}. Розроблений алгоритм можна використовувати для ефективного вирішення задачі стерео зору в умовах повільного поступового надходження даних.
	
