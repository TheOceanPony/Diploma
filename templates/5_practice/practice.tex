\chapter{Практичні результати}\label{chapter4}

У четвертому розділі показані практичні результати роботи алгоритмів запропонованих в розділі \ref{chapter3}. 
Порівнюється час роботи запропонованих алгоритмів з часом роботи алгоритмів, наведених в розділі \ref{chapter2}.   

\section{Умови тестування}
Всі алгоритми були реалізовані на мові  \textbf{C++} , з використанням бібліотеки \textbf{OpenCV}, для зручної роботи з зображеннями.

Алгоритми тестувалися на зображеннях з \textbf{Middlebury Stereo Datasets}, з $ D_{max} = 100 $. В якості штрафів були вибрані
\begin{align}
 h(i,d) &= | \mathcal{L}(i) - \mathcal{R}(i-d) |,\\
 g(d, d') &= \alpha | d  - d'|,
\end{align}
де $ \alpha $ -- коефіцієнт згладжування, що підбирався експериментально.

\section{Поступове упорядковане надходження даних}

\begin{figure}
\begin{subfigure}{.5\textwidth}
  \centering
  \includegraphics[width=.8\linewidth]{view1}
  \caption{1a}
  \label{fig:sfig1}
\end{subfigure}%
\begin{subfigure}{.5\textwidth}
  \centering
  \includegraphics[width=.8\linewidth]{view2}
  \caption{1b}
  \label{fig:sfig2}
\end{subfigure}
\caption{plots of....}
\label{fig:fig}
\end{figure}

\section{Неупорядковане надходження даних при одному відомому зображені}
